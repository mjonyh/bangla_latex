% !TEX TS-program = XeLaTeX

% This document is written as a template of Bangla latex
% Author: Md. Enamul Hoque
% Affiliation: Department of Physics, SUST
% email: mjonyh@gmail.com

\documentclass[12pt]{report}
\usepackage{polyglossia}

\setmainlanguage[numerals=Devanagari]{bengali}
\setotherlanguage{english}
\newfontfamily\englishfont[Scale=MatchLowercase]{Monaco}
\newfontfamily\bengalifont[Script=Bengali]{SolaimanLipi}

\setmainfont[Script=Devanagari]{SolaimanLipi}
\makeatletter
\def\bengalidigits#1{\expandafter\@bengali@digits #1@}
\def\@bengali@digits#1{%
  \ifx @#1
  \else
    \ifx0#1০\else\ifx1#1১\else\ifx2#1২\else\ifx3#1৩\else\ifx4#1৪\else\ifx5#1৫\else\ifx6#1৬\else\ifx7#1৭\else\ifx8#1৮\else\ifx9#1৯\fi\fi\fi\fi\fi\fi\fi\fi\fi\fi
    \expandafter\@bengali@digits
  \fi
}
\makeatother

\def\bengalinumber#1{\bengalidigits{\number#1}}
\def\bengalinumeral#1{\bengalinumber{\csname c@#1\endcsname}}

\renewcommand\thepart{\bengalinumeral{part}}
\renewcommand\thechapter{\bengalinumeral{chapter}}
\renewcommand\thesection{\bengalinumeral{section}}
\renewcommand\thesubsection{\bengalinumeral{subsection}}
\renewcommand\thesubsubsection{\bengalinumeral{subsubsection}}
\renewcommand\thepage{\bengalinumeral{page}}
\renewcommand\theenumi{\bengalinumeral{enumi}}

\usepackage{fontspec}

%%%%%%%%%%%%%%%%%%%%%%%%%%%% begining of the document %%%%%%%%%%%%%%%%%%%%%%%%%%%%%
\begin{document}

%%%%%%%%%%%%-------------- Title page ------------------ %%%%%%%%%%%%%%%%%%%%%%%%%%
\begin{titlepage}
	\centering
	\vspace{1.5cm}
	{\huge\bfseries সাধারণ ডায়েরি\par}
	\vspace{1.25cm}
	{\Large\itshape মুঃ এনামুল হক\par}
	\vfill

	{\large \today\par}
\end{titlepage}
%%%%%%%%%%%%-------------- End of Title page ----------- %%%%%%%%%%%%%%%%%%%%%%%%%%

\tableofcontents

\part{দিনের শুরু}
\chapter{পটভূমি}
\section{ভূমিকা}
এটি একটি ইংরেজি লেখা ছোট অধ্যায় যে বাঙ্গালী এ \textenglish{Google Translate} দ্বারা অনুবাদ করা হয়েছে. এটা খুব স্পষ্ট নয় যদি সঠিক অনুবাদ বা না কিন্তু ক্রিয়াটি ফন্ট দেখাতে যথেষ্ট হওয়া উচিত.

\begin{enumerate}
    \item   \foreignlanguage{english}{This is first}
    \item   বাংলা লেখা
\end{enumerate}

\section{বৃহস্পতিবার, ২৬ জানুয়ারি, ২০১৭}
ঘুম থেকে চোখ খুলে দেখি ঘড়িতে সকাল ৯.২০। আজ আমার সকাল ৯ টায় অফিসে থাকার কথা। তাই পড়িমরি করে অফিসের জন্য তৈরি হয়ে বের পড়লাম আমার সাইকেলটি নিয়ে।

\subsection{প্রথম সাবসেকশন}
\subsection{দ্বিতীয় সাবসেকশন}
‌‌\subsubsection{প্রথম সাবসাবসেকশন}

\end{document}
